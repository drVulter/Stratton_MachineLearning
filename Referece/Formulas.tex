
\documentclass{article}
\usepackage[utf8]{inputenc}
\usepackage{amsthm}
\usepackage{amsmath, mathtools, amsfonts,amssymb}
\usepackage{graphicx}
\usepackage{verbatim}
\usepackage{fancyhdr}
% Code
\usepackage{listings} 
\usepackage{algorithm}
\usepackage{algpseudocode}
% Margins
\usepackage{geometry}
%\usepackage{enumitem} % for nice enumerating
\geometry{margin=1in}

% Graphics
\usepackage{tikz}
\usetikzlibrary{matrix} % matrices
\usepackage{tikz-qtree} % Simple trees
\usepackage{verbatim} % what is this for again???

\setcounter{section}{-1}

\newtheorem{pic}{Figure}
\numberwithin{pic}{section}
\newtheorem{lem}{Lemma}
\numberwithin{lem}{section}
\newtheorem{thm}{Theorem}
\numberwithin{thm}{section}
\newtheorem{cor}{Corollary}
\numberwithin{cor}{section}

\theoremstyle{definition}
\newtheorem{ex}{Example}
\numberwithin{ex}{section}
\newtheorem{defn}{Definition}
\numberwithin{defn}{section}
\theoremstyle{definition}
\newtheorem{prob}{Problem}

\theoremstyle{remark}
\newtheorem*{con}{Conjecture}
\newtheorem{rem}{Remark}
\newtheorem*{cex}{Counterexample}
\newtheorem*{ts}{T.S.}

%%% COMMANDS %%%
% Sets
\newcommand{\set}[1]{\ensuremath{\left\{ #1\right\}}} % write sets
\newcommand{\e}{\ensuremath{\epsilon}} % Epsilon
\newcommand{\R}{\ensuremath{\mathbb{R}}} % Real Numbers
\newcommand{\N}{\ensuremath{\mathbb{N}}} % Natural numbers
\newcommand{\Q}{\ensuremath{\mathbb{Q}}} % Rationals
\newcommand{\I}{\ensuremath{\mathbb{I}}} % Irrational Numbers
\newcommand{\Z}{\ensuremath{\mathbb{Z}}} % Integers
% Easier Delimiters?
\newcommand{\lr}[2]{\ensuremath{\left#1 #2 \right #1}}
% Absolute Value
\newcommand{\abs}[1]{\ensuremath{\left| #1 \right|}}
% Landau Notation
\newcommand{\Oh}{\ensuremath{\mathcal{O}}} %%% IN MATH MODE
\newcommand{\oh}{\ensuremath{\mathcal{o}}} %%% IN MATH MODE
% Display style fractions
\newcommand{\Frac}[2]{\displaystyle \frac{#1}{#2}}
% Display style limits
\newcommand{\Lim}[2]{\displaystyle \lim_{#1}{#2}}

% Enumerate
\renewcommand{\labelenumi}{(\alph{enumi})}
\renewcommand{\labelenumii}{\roman{enumii}}

% change proof environment
\renewcommand*{\proofname}{Pf}

% Indentation
\newlength\tindent
\setlength{\tindent}{\parindent}
\setlength{\parindent}{0pt}
\renewcommand{\indent}{\hspace*{\tindent}}

% Set title
\title{Formulas}

\begin{document}


\fancyhead[l]{Quinn Stratton}
\fancyhead[c]{ML Formulas}
\fancyhead[r]{\today}
\pagestyle{fancy}
\section{Formulas}
\begin{align}
  Precision &= \frac{TP}{TP + FP}\\
  Recall &= \frac{TP}{TP + FN}\\
  F_{score} &= \frac{2TP}{2TP + FN + FP}\\
  F_{score (weighted)} &= \frac{(1+\beta)TP}{(1+\beta)TP + \beta + FN + FP}
\end{align}
If $\beta = 0$, $F_{score (weighted)} = \Frac{TP}{TP + FP}$
so \textit{Precision} is emphasized.\\
For $n$ subgroups, the \textbf{Macro Average} is the average of the subgroups.
\begin{align}
  \text{Macro Average} &= \frac{\sum_{i=1}^n \text{Evaluation}_i}{n}
\end{align}
%If $\beta = 1$, $F_{score (weighted)} = \Frac{TP}{TP + FP}$
%so \textit{Precision} is emphasized.
\section{Terms}
\begin{defn}
  \textbf{False Positive}: Looks like a winner, but is not.
\end{defn}

\begin{defn}
  \textbf{False Negative}: Result is true, but predicted to be false.
\end{defn}

\begin{defn}
  \textbf{Precision}: Emphasizes \textit{correct} results. Consider a Google Search.
\end{defn}

\begin{defn}
  \textbf{Recall}: ``Better safe than sorry''. Emphasizes more positives, even if incorrect since they can be checked later. Consider a cancer screening.
\end{defn}

\begin{defn}
  \textbf{Overfitting}: Model fits training data, but does not show overall trend, i.e. does not accurately predict labels for test data.
\end{defn}

\begin{defn}
  \textbf{Underfitting}: Model does not fit training data, and is too general for an accurate prediction.
\end{defn}



\end{document}